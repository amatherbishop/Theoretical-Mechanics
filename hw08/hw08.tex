\documentclass[11pt,letterpaper,boxed]{../hmcpsetrhino}
\usepackage[margin=1in]{geometry}
\usepackage{graphicx}
\usepackage{enumerate}
\usepackage{amsthm}
\usepackage{amsmath}

\newcommand{\ds}{\displaystyle}
\newcommand{\half}{\frac{1}{2}}
\newcommand*\Eval[3]{\left.#1\right\rvert_{#2}^{#3}}
\newcommand{\eval}{\biggr\rvert}
\newcommand\Partial[2]{\frac{\partial #1}{\partial #2}}
\let\oldvec\vec
\renewcommand{\vec}[1]{\oldvec{\mathbf{#1}}}
\def\EE{{\cal E}}
\def\Lagr{\mathcal{L}}
\def\Ham{\mathcal{H}}

\name{}
\class{Physics 111 Section 1}
\assignment{Problem Set 08}
\duedate{October 5, 2016}

\begin{document}

\problemlist{Central Forces: Kepler Orbits}
\textbf{Help:} 

\begin{problem}[i]
Convince yourself once and for all of this surprising and historically important result: particles in bound Kepler orbits follow an elliptical path with the sun at a focus. Your ancestors spent lifetimes trying to understand the celestial bodies, and how you are among the select few who do.
\begin{problem}[8.16]
We have proved in (8.49) that any Kepler orbit can be written in the form $r(\phi) = c/1 + \epsilon \cos \phi)$, where $c>0$ and $\epsilon \geq 0$. For the case that $0 \leq \epsilon <1$, rewrite this equation in rectangular coordinates $(x,y)$ and prove that the equation can be cast in the form (8.51), which is the equation of an ellipse. Verify the values of the constants given in (8.52).\\
(8.49):
\[	r(\phi) = \frac{c}{1 + \epsilon \cos \phi}\]
(8.51):
\[	\frac{(x+d)^2}{a^2} + \frac{y^2}{b^2} = 1\]
(8.52): 
\[	a = \frac{c}{1 - \epsilon^2}, \qquad b = \frac{c}{\sqrt{1 - \epsilon^2}}, \qquad \text{and} \  d = a \epsilon\]
\end{problem}
\vspace{-0.45cm}
\end{problem}
\begin{solution}


\vfill
\end{solution}

\newpage 

\begin{problem}[ii]
Now convince yourself once and for all that particles in unbound Kepler orbits follow parabolic or hyperbolic trajectories.\\
\begin{problem}[8.30]
The general Kepler orbit is given in polar coordinates by (8.49). Rewrite this in Cartesian coordinates for the cases that $\epsilon = 1$ and $\epsilon > 1$. Show that if $\epsilon = 1$, you get the parabola (8.60), and if $\epsilon >1$, the hyperbola (8.61). For the latter, identify the constants $\alpha$, $\beta$, and $\delta$ in terms of $c$ and $\epsilon$.\\

(8.60):
\[	y^2 =c^2 - 2cx\]
(8.61):
\[	\frac{(x-\delta)^2}{\alpha^2} - \frac{y^2}{\beta^2} = 1\]
\end{problem}
\vspace{-0.45cm}
\end{problem}
\begin{solution}


\vfill
\end{solution}


\newpage

\begin{problem}[iii]
A repulsive inverse-square law\\
\begin{problem}[8.13]
Consider the motion of two particles subject to a \textit{repulsive} inverse-square force (for example, two positive charges). Show that this system has no states with  $E < 0$ (as measured in the CM frame), and that in all states with $E > 0$, the relative motion follows a hyperbola. Sketch a typical orbit. [\textit{Hint}: You can follow closely the analysis of Sections 8.6 and 8.7 except that you must reverse the force; probably the simplest way to do this is to change the sign of $\gamma$ in (8.44) and all subsequent equations (so that $F(r) = +\gamma/r^2$) and then keep $\gamma$ itself positive. Assume $\ell \neq 0$.]
\end{problem}
\vspace{-0.45cm}
\end{problem}

\begin{solution}

\vfill
\end{solution}

\end{document}
