\documentclass[11pt,letterpaper,boxed]{../hmcpset}
\usepackage[margin=1in]{geometry}
\usepackage{graphicx}
\usepackage{enumerate}
\usepackage{amsthm}
\usepackage{amsmath}

\newcommand{\ds}{\displaystyle}
\newcommand{\half}{\frac{1}{2}}
\newcommand*\Eval[3]{\left.#1\right\rvert_{#2}^{#3}}
\newcommand{\eval}{\biggr\rvert}
\newcommand\Partial[2]{\frac{\partial #1}{\partial #2}}
\renewcommand{\vec}[1]{\mathbf{#1}}
\def\EE{{\cal E}}
\def\Lagr{\mathcal{L}}


\name{}
\class{Physics 111 Section 1}
\assignment{Problem Set 04}
\duedate{September 14, 2016}

\begin{document}

\problemlist{The Lagrangian, Noether's Theorem \& the Hamiltonian (Reading: Chapter 7.6 - 7.8)}
\textbf{Help:} 

\begin{problem}[i]
Review the discussion of the Hamiltonian and total energy found on pp. 269-272 of Taylor. Then repeat the arguments from eq. 7.92 to eq. 7.98 to show that the Hamiltonian is the total energy, $\mathcal{H} = T + U$, for systems in which the transformation between generalized coordinates and Cartesian coordinates is time-independent (i.e., the generalized coordinates are "natural"). You may look at the textbook as necessary while you re-work this proof, but keep in mind that the point of this exercise is to really understand the argument, not just reproduce the work on the textbook page.
\end{problem}

\begin{solution}

\vfill
\end{solution}

\newpage 

\begin{problem}[ii]
In section 7.8 Taylor showed that Lagrangians which are invariant to spatial translation imply conservation of linear momentum, while Lagrangians which are invariant to time translation imply conservation of energy (with some qualifications). I this problem you will show that Lagrangians which re invariant under rotation imply conservation of angular momentum. Take a moment to contemplate these results: Symmetries and conservation laws are deeply related - an insight that comes to us from Emmi Noether and our Lagrangian formalism.

\begin{problem}[7.46]
Noether's theorem asserts a connection between invariance principles and conservation laws. In Section 7.8 we saw that translational invariance of the Lagrangian implies conservation of total linear momentum. Here you will prove that rotational invariance of $\Lagr$ implies conservation of total angular momentum. Suppose that the Lagrangian of an $N$-particle system is unchanged by rotations about a certain symmetry axis.
\begin{enumerate}[(a)]
\item Without loss of generality, take this axis to be the $z$ axis, and show that the Lagrangian is unchanged when all of the particles are simultaneously moved from $r_\alpha, \theta_\alpha, \phi_\alpha)$ to $(r_\alpha, \theta_\alpha, \phi_\alpha + \epsilon)$ (same $\epsilon$ for all particles). Hence show that 
\[	\sum_{\alpha = 1}^N \frac{\partial \Lagr}{\partial \phi_\alpha} = 0\]
\item Use Lagrange's equations to show that this implies that the total angular momentum $L_z$ about the symmetry axis is constant. In particular, if the Lagrangian is invariant under rotations about all axes, then all components of $\vec L$ are conserved.

\end{enumerate}
\end{problem}
\end{problem}

\begin{solution}

\vfill
\end{solution}

\end{document}
