\documentclass[11pt,letterpaper,boxed]{../hmcpsetrhino}
\usepackage[margin=1in]{geometry}
\usepackage{graphicx}
\usepackage{enumerate}
\usepackage{amsthm}
\usepackage{amsmath}

\newcommand{\ds}{\displaystyle}
\newcommand{\half}{\frac{1}{2}}
\newcommand*\Eval[3]{\left.#1\right\rvert_{#2}^{#3}}
\newcommand{\eval}{\biggr\rvert}
\newcommand\Partial[2]{\frac{\partial #1}{\partial #2}}
\newcommand\mat[1]{\underline{\vec {#1}}}
\newcommand\tp[1]{\widetilde {#1}}
\def\EE{{\cal E}}
\def\Lagr{\mathcal{L}}
\def\Ham{\mathcal{H}}

\name{}
\class{Physics 111 Section 1}
\assignment{Problem Set 13}
\duedate{October 31, 2016}

\begin{document}

\problemlist{Rotational Motion: Euler's Equations}
\textbf{Help:}

\begin{problem}[i]
Angular momentum and kinetic energy in torque-free rotation

\begin{problem}[10.40]
\begin{enumerate}[(a)]
\item A rigid body is rotating freely, subject to zero torque. Use Euler's equations (10.88) to prove that the magnitude of the angular momentum $\vec L$ is constant. (Multiply the $i$th equation by $L_i = \lambda_i \omega_i$ and add the three equations.) 
\item In much the same way, show that the kinetic energy of rotation $T_{\text{rot}} = \half (\lambda_1 {\omega_1}^2 + \lambda_2 {\omega_2} ^2 +\lambda_3 {\omega_3}^2)$, as in (10.68), is constant.


\end{enumerate}
\end{problem}
\end{problem}
\begin{solution}


\vfill
\end{solution}

\newpage 

\begin{problem}[ii]
An accelerating, rotating space station.

\begin{problem}[10.44]
An axially symmetric space station (principal axis $\vec e_3$, and $\lambda_1 = \lambda_2$) is floating in free space. It has rockets mounted symmetrically on either side that are firing and exert a constant torque $\Gamma$ about the symmetry axis. Solve Euler's equations exactly for $\vec \omega$ (relative to the body axis) and describe the motion. At $t=0$ take $\vec \omega = (\omega_{10}, 0 , \omega_{30})$.

\end{problem}
\end{problem}
\begin{solution}


\vfill
\end{solution}

\newpage 

\begin{problem}[iii]
Rate of precession in the space frame, $\Omega_s$.

\begin{problem}[10.46]
 We saw in Section 10.8 that in the free precession of an axially symmetric body the three vectors $\vec e_3$ (the body axis), $\vec \omega$, and $\vec L$ lie in a plane. As seen in the body frame, $\vec e_3$ is fixed, and $\vec \omega$ and $\vec L$ precess around $\vec e_3$ with angular velocity $\Omega_b = \omega_3 (\lambda_1 - \lambda_3)/\lambda_1$. As seen in the space frame $\vec L$ is fixed and $\vec \omega$ and $\vec e_3$ precess around $\vec L$ with angular velocity $\Omega_s$. In this problem you will find three equivalent expressions for $\Omega_s$. 
 \begin{enumerate}[(a)]
 \item Argue that $\vec \Omega_s = \vec \Omega_b + \vec \omega$. [Remember that relative angular velocities add like vectors.]
 
 \item Bearing in mind that $\vec \Omega_b$ is parallel to $\vec e_3$ prove that $\Omega_s = \omega \sin \alpha / \sin \theta$ where $\alpha$ is the angle between $\vec e_3$ and $\vec \omega$ and $\theta$ is that between $\vec e_3$ and $\vec L$ (see Figure 10.9). 
 
 \item Thence prove that
 \[	\Omega_s = \omega \frac{\sin \alpha}{\sin \theta} = \frac{L}{\lambda_1} = \omega \frac{\sqrt{{\lambda_3}^2 + ({\lambda_1}^2 - {\lambda_3}^2) \sin^2 \alpha}}{\lambda_1}\]
 
 \end{enumerate}

\end{problem}
\end{problem}
\begin{solution}


\vfill
\end{solution}

\newpage 

\begin{problem}[iv]
Chandler wobble on a hypothetical Earth.

\begin{problem}[10.47]
Imagine that this world is perfectly rigid, uniform, and spherical and is spinning about sits usual axis at its usual rate. A huge mountain of mass $10^{-8}$ earth masses is now added at colatitude 60$^\circ$, causing the earth to begin the free precession described in Section 10.8. How long will it take the North Pole (defined as the northern end of the diameter along $\vec \omega$) to move 100 miles from its current position? [Take the earth's radius to be 4000 miles.]

\end{problem}
\end{problem}
\begin{solution}


\vfill
\end{solution}


\end{document}
