\documentclass[11pt,letterpaper,boxed]{../hmcpsetrhino}
\usepackage[margin=1in]{geometry}
\usepackage{graphicx}
\usepackage{enumerate}
\usepackage{amsthm}
\usepackage{amsmath}

\newcommand{\ds}{\displaystyle}
\newcommand{\half}{\frac{1}{2}}
\newcommand*\Eval[3]{\left.#1\right\rvert_{#2}^{#3}}
\newcommand{\eval}{\biggr\rvert}
\newcommand\Partial[2]{\frac{\partial #1}{\partial #2}}
\newcommand\mat[1]{\underline{\vec {#1}}}
\newcommand\tp[1]{\widetilde {#1}}
\def\EE{{\cal E}}
\def\Lagr{\mathcal{L}}
\def\Ham{\mathcal{H}}

\name{}
\class{Physics 111 Section 1}
\assignment{Problem Set 14}
\duedate{November 2, 2016}

\begin{document}

\problemlist{Rotational Motion: Spinning Top (Reading: Chapter 10.9 - 10.10)}
\textbf{Help:}

\begin{problem}[i]
Comparing components of angular momentum in the space frame and body frame. More vegetables.

\begin{problem}[10.49]
Starting from Equation (10.100) for $\vec L$, verify that $L_z$ is correctly given by Equations (10.102) and (10.103).\\
\[	\vec L = (-\lambda_1 \dot \phi \sin \theta) \vec e_1'  + \lambda_1 \dot \theta \vec e_2' + \lambda_3 (\dot \psi + \dot \phi \cos \theta) \vec e_3 \tag{10.100}\]
\begin{align*}
L_z &= \lambda_1 \dot \phi \sin^2 \theta + \lambda_3(\dot \psi + \dot \phi \cos \theta) \cos \theta \tag{10.102}\\
&= \lambda_1 \dot \phi \sin^2 \theta + L_3 \cos \theta \tag{10.103}
\end{align*}
\end{problem}
\end{problem}
\begin{solution}


\vfill
\end{solution}

\newpage 

\begin{problem}[ii]
Chandler wobble for a symmetric top.

\begin{problem}[10.52]
Consider the rapid steady precession of a symmetric top predicted in connection with (10.112).
\begin{enumerate}[(a)]
\item Show that in this motion the angular momentum $\vec L$ must be very close to the vertical. [\textit{Hint}: Use (10.100) to write down the horizontal component $L_{\text{hor}}$ of $\vec L$. Show that if $\dot \phi$ is given by the right side of (10.112), $L_{\text{hor}}$ is exactly zero.]

\item Use this result to show that the rate of precession $\Omega$ given in (10.112) agrees with the free precession rate $\Omega_s$ found in (10.96).

\end{enumerate}
\end{problem}
\end{problem}
\begin{solution}


\vfill
\end{solution}

\end{document}
