\documentclass[11pt,letterpaper,boxed]{../hmcpsetrhino}
\usepackage[margin=1in]{geometry}
\usepackage{graphicx}
\usepackage{enumerate}
\usepackage{amsthm}
\usepackage{amsmath}

\newcommand{\ds}{\displaystyle}
\newcommand{\half}{\frac{1}{2}}
\newcommand*\Eval[3]{\left.#1\right\rvert_{#2}^{#3}}
\newcommand{\eval}{\biggr\rvert}
\newcommand\pd[2]{\frac{\partial #1}{\partial #2}}
\newcommand\mat[1]{\underline{\vec {#1}}}
\newcommand\tp[1]{\widetilde {#1}}
\def\EE{{\cal E}}
\def\Lagr{\mathcal{L}}
\def\Ham{\mathcal{H}}

\name{}
\class{Physics 111 Section 1}
\assignment{Problem Set 20}
\duedate{December 7, 2016}

\begin{document}

\problemlist{Hamiltonian Mechanics - Phase Space (Reading: Chapter 13.5 - 13.6)}
\textbf{Help:}

\begin{problem}[i]
A canonical transformation of the simple harmonic oscillator Hamiltonian.

\begin{problem}[13.25]
Here is another example of the canonical transformation, which is still too simple to be of any real use, but does nevertheless illustrate the power of these changes of coordinates.

\begin{enumerate}[(a)]
\item Consider a system with one degree of freedom and Hamiltonian $\Ham = \Ham(q,p)$ and a new pair of coordinates $Q$ and $P$ defined so that
\[	q = \sqrt{2P} \sin Q \qquad \text{and} \qquad p = \sqrt{2P} \cos Q\]
Prove that if $\pd \Ham q = -\dot p$ and $\pd \Ham p = \dot q$, it automatically follows that $\pd \Ham Q = -\dot P$ and $\pd \Ham P = \dot Q$. In other words, the Hamiltonian formalism applies just as well to the new coordinates as to the old. 
\item Show that the Hamiltonian of a one-dimensional harmonica oscillator with mass $m = 1$ and force constant $k = 1$ is $\Ham = \half( q^2 + p^2)$.

\item Show that if you rewrite this Hamiltonian in terms of the coordinates $Q$ and $P$ defined in, then $Q$ is ignorable. [The change of coordinates was cunningly chosen to produce this elegant result.] What is $P$? 

\item Solve the Hamiltonian equation for $Q(t)$ and verify that, when rewritten for $q$, your solution give the expected behavior.
\end{enumerate}
\end{problem}
\end{problem}
\begin{solution}


\vfill
\end{solution}


\newpage

\begin{problem}[ii]
Phase-space orbits.

\begin{problem}[13.28]
Consider a mass $m$ confined to the $x$ axis and subject to a force $F_x = kx$ where $k > 0$. 
\begin{enumerate}[(a)]
\item Write down and sketch the potential energy $U(x)$ and describe the possible motions of the mass. (Distinguish between the cases that $E>0$ and $E<0$.)

\item Write down the Hamiltonian $\Ham (x,p)$, and describe the possible phase-space orbits for the two cases $E> 0$ and $E<0$. (Remember that the function $\Ham (x,p)$ must equal the constant energy $E$.) Explain your answers to part (b) in terms of those  to part (a).

\end{enumerate}
\end{problem}
\end{problem}
\begin{solution}


\vfill
\end{solution}

\end{document}
