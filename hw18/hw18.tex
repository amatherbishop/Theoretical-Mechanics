\documentclass[11pt,letterpaper,boxed]{../hmcpsetrhino}
\usepackage[margin=1in]{geometry}
\usepackage{graphicx}
\usepackage{enumerate}
\usepackage{amsthm}
\usepackage{amsmath}

\newcommand{\ds}{\displaystyle}
\newcommand{\half}{\frac{1}{2}}
\newcommand*\Eval[3]{\left.#1\right\rvert_{#2}^{#3}}
\newcommand{\eval}{\biggr\rvert}
\newcommand\Partial[2]{\frac{\partial #1}{\partial #2}}
\newcommand\mat[1]{\underline{\vec {#1}}}
\newcommand\tp[1]{\widetilde {#1}}
\def\EE{{\cal E}}
\def\Lagr{\mathcal{L}}
\def\Ham{\mathcal{H}}

\name{}
\class{Physics 111 Section 1}
\assignment{Problem Set 18}
\duedate{November 30, 2016}

\begin{document}

\problemlist{Coupled Oscillations - CO$_2$ (Reading: Chapter 11.5 - 11.6)}
\textbf{Help:}

\begin{problem}[i]
Reproduce the arguments found in the Section 1 of Prof. Saeta's handout entitled \textit{Generalizations to N Coupled Particles} to arrive at Lagrange's equations and the characteristic equation for $N$ masses which are coupled by springs and have a stable equilibrium configuration. \textbf{OR}, you may choose to reproduce the arguments of Taylor's section 11.5, \textit{The General Case}, which covers the same material and arrives at the same characteristic equation. (I prefer Prof. Saeta's discussion of the material, but you may prefer Taylor's. Either way, they complement each other.) In you write-up please pay particular attention to how you arrive at both $A_{jk}$ and $m_{jk}$ and what assumptions are being made throughout the derivation.\\

As always the point of this exercise is to make sure that you understand the arguments, not just blast through it to get finished. You may find it helpful to re-read Taylor's section 11.5 (\textit{The General Case}) and 7.8 (\textit{More about Conservation Laws}) to better understand the arguments being made.

\end{problem}
\begin{solution}


\vfill
\end{solution}

\end{document}
